
% This LaTeX was auto-generated from an M-file by MATLAB.
% To make changes, update the M-file and republish this document.

%%% \documentclass{article}
%%% \usepackage{graphicx}
%%% \usepackage{color}

%%% \sloppy
%%% \definecolor{lightgray}{gray}{0.5}
\setlength{\parindent}{0pt}

%%% \begin{document}

    
    
\subsection*{Example of the QWTB use}

\begin{par}
Data are simulated, QWTB is used with different algorithms.
\end{par} \vspace{1em}

\subsubsection*{Contents}

\begin{itemize}
\setlength{\itemsep}{-1ex}
   \item Generate ideal data
   \item Apply three algorithms
   \item Compare results for ideal signal
   \item Noisy signal
   \item Compare results for noisy signal
   \item Non-coherent signal
   \item Compare results for non-coherent signal
   \item Harmonically distorted signal.
   \item Compare results for harmonically distorted signal.
   \item Harmonically distorted, noisy, non-coherent signal.
   \item Compare results for harmonically distorted, noisy, non-coherent signal.
\end{itemize}


\subsubsection*{Generate ideal data}

\begin{par}
Sample data are generated, representing 1 second of sine waveform of nominal frequency \lstinline{fnom} 1000 Hz, nominal amplitude \lstinline{Anom} 1 V and nominal phase \lstinline{phnom} 1 rad. Data are sampled at sampling frequency \lstinline{fsnom} 10 kHz, perfectly synchronized, no noise.
\end{par} \vspace{1em}
\begin{lstlisting}[style=mcode]
Anom = 1; fnom = 1000; phnom = 1; fsnom = 10e4;
timestamps = [0:1/fsnom:0.1-1/fsnom];
ideal_wave = Anom*sin(2*pi*fnom*timestamps + phnom);
\end{lstlisting}
\begin{par}
To use QWTB, data are put into two quantities: \lstinline{t} and \lstinline{y}. Both quantities are put into data in structure \lstinline{DI}.
\end{par} \vspace{1em}
\begin{lstlisting}[style=mcode]
DI = [];
DI.t.v = timestamps;
DI.y.v = ideal_wave;
\end{lstlisting}


\subsubsection*{Apply three algorithms}

\begin{par}
QWTB will be used to apply three algorithms to determine frequency and amplitude: \lstinline{SP-FFT}, \lstinline{PSFE} and \lstinline{FPSWF}. Results are in data out structure \lstinline{DOxxx}. Algorithm \lstinline{FPSWF} requires an estimate, select it to 0.1\% different from nominal frequency. \lstinline{SP-FFT} requires sampling frequency.
\end{par} \vspace{1em}
\begin{lstlisting}[style=mcode]
DI.f.v = fnom.*1.001;
DI.fs.v = fsnom;
DOspfft = qwtb('SP-FFT', DI);
DOpsfe = qwtb('PSFE', DI);
DOfpswf = qwtb('FPSWF', DI);
\end{lstlisting}

        \begin{lstlisting}[style=output]
QWTB: no uncertainty calculation
QWTB: no uncertainty calculation
QWTB: no uncertainty calculation
Fitting started

Local minimum found.

Optimization completed because the size of the gradient is less than
the default value of the function tolerance.



Fitting finished
\end{lstlisting} \color{black}
    

\subsubsection*{Compare results for ideal signal}

\begin{par}
Calculate relative errors in ppm for all algorithm to know which one is best. \lstinline{SP-FFT} returns whole spectrum, so only the largest amplitude peak is interesting. One can see for the ideal case all errors are very small.
\end{par} \vspace{1em}
\begin{lstlisting}[style=mcode]
disp('SP-FFT errors (ppm):')
[tmp, ind] = max(DOspfft.A.v);
ferr  = (DOspfft.f.v(ind) - fnom)/fnom .* 1e6
Aerr  = (DOspfft.A.v(ind) - Anom)/Anom .* 1e6
pherr = (DOspfft.ph.v(ind) - phnom)/phnom .* 1e6

disp('PSFE errors (ppm):')
ferr  = (DOpsfe.f.v - fnom)/fnom .* 1e6
Aerr  = (DOpsfe.A.v - Anom)/Anom .* 1e6
pherr = (DOpsfe.ph.v - phnom)/phnom .* 1e6

disp('FPSWF errors (ppm):')
ferr  = (DOfpswf.f.v - fnom)/fnom .* 1e6
Aerr  = (DOfpswf.A.v - Anom)/Anom .* 1e6
pherr = (DOfpswf.ph.v - phnom)/phnom .* 1e6
\end{lstlisting}

        \begin{lstlisting}[style=output]
SP-FFT errors (ppm):

ferr =

     0


Aerr =

     0


pherr =

  -1.5708e+06

PSFE errors (ppm):

ferr =

  -2.2737e-10


Aerr =

   4.8850e-09


pherr =

   2.3093e-08

FPSWF errors (ppm):

ferr =

  -3.4106e-10


Aerr =

  -4.0479e-07


pherr =

   3.7526e-08

\end{lstlisting} \color{black}
    

\subsubsection*{Noisy signal}

\begin{par}
To simulate real measurement, noise is added with normal distribution and standard deviation \lstinline{sigma} of 100 microvolt. Algorithms are again applied.
\end{par} \vspace{1em}
\begin{lstlisting}[style=mcode]
sigma = 100e-6;
DI.y.v = ideal_wave + normrnd(0, 100e-6, size(ideal_wave));
DOspfft = qwtb('SP-FFT', DI);
DOpsfe = qwtb('PSFE', DI);
DOfpswf = qwtb('FPSWF', DI);
\end{lstlisting}

        \begin{lstlisting}[style=output]
QWTB: no uncertainty calculation
QWTB: no uncertainty calculation
QWTB: no uncertainty calculation
Fitting started

Local minimum found.

Optimization completed because the size of the gradient is less than
the default value of the function tolerance.



Fitting finished
\end{lstlisting} \color{black}
    

\subsubsection*{Compare results for noisy signal}

\begin{par}
Again relative errors are compared. One can see amplitude and phase errors increased to several ppm, however frequency is still determined quite good by all three algorithms. FFT is not affected by noise at all.
\end{par} \vspace{1em}
\begin{lstlisting}[style=mcode]
disp('SP-FFT errors (ppm):')
[tmp, ind] = max(DOspfft.A.v);
ferr  = (DOspfft.f.v(ind) - fnom)/fnom .* 1e6
Aerr  = (DOspfft.A.v(ind) - Anom)/Anom .* 1e6
pherr = (DOspfft.ph.v(ind) - phnom)/phnom .* 1e6

disp('PSFE errors:')
ferr  = (DOpsfe.f.v - fnom)/fnom .* 1e6
Aerr  = (DOpsfe.A.v - Anom)/Anom .* 1e6
pherr = (DOpsfe.ph.v - phnom)/phnom .* 1e6

disp('FPSWF errors:')
ferr  = (DOfpswf.f.v - fnom)/fnom .* 1e6
Aerr  = (DOfpswf.A.v - Anom)/Anom .* 1e6
pherr = (DOfpswf.ph.v - phnom)/phnom .* 1e6
\end{lstlisting}

        \begin{lstlisting}[style=output]
SP-FFT errors (ppm):

ferr =

     0


Aerr =

   -0.8795


pherr =

  -1.5708e+06

PSFE errors:

ferr =

   -0.0035


Aerr =

   -0.9103


pherr =

    3.0107

FPSWF errors:

ferr =

   -0.0087


Aerr =

   -0.8780


pherr =

    4.4147

\end{lstlisting} \color{black}
    

\subsubsection*{Non-coherent signal}

\begin{par}
In real measurement coherent measurement does not exist. So in next test the frequency of the signal differs by 20 ppm:
\end{par} \vspace{1em}
\begin{lstlisting}[style=mcode]
fnc = fnom*(1 + 20e-6);
noncoh_wave = Anom*sin(2*pi*fnc*timestamps + phnom);
DI.y.v = noncoh_wave;
DOspfft = qwtb('SP-FFT', DI);
DOpsfe = qwtb('PSFE', DI);
DOfpswf = qwtb('FPSWF', DI);
\end{lstlisting}

        \begin{lstlisting}[style=output]
QWTB: no uncertainty calculation
QWTB: no uncertainty calculation
QWTB: no uncertainty calculation
Fitting started

Local minimum found.

Optimization completed because the size of the gradient is less than
the default value of the function tolerance.



Fitting finished
\end{lstlisting} \color{black}
    

\subsubsection*{Compare results for non-coherent signal}

\begin{par}
Comparison of relative errors. Results of PSFE or FPSWF are correct, however FFT is affected by non-coherent signal considerably.
\end{par} \vspace{1em}
\begin{lstlisting}[style=mcode]
disp('SP-FFT errors (ppm):')
[tmp, ind] = max(DOspfft.A.v);
ferr  = (DOspfft.f.v(ind) - fnc)/fnc .* 1e6
Aerr  = (DOspfft.A.v(ind) - Anom)/Anom .* 1e6
pherr = (DOspfft.ph.v(ind) - phnom)/phnom .* 1e6

disp('PSFE errors:')
ferr  = (DOpsfe.f.v - fnc)/fnc .* 1e6
Aerr  = (DOpsfe.A.v - Anom)/Anom .* 1e6
pherr = (DOpsfe.ph.v - phnom)/phnom .* 1e6

disp('FPSWF errors:')
ferr  = (DOfpswf.f.v - fnc)/fnc .* 1e6
Aerr  = (DOfpswf.A.v - Anom)/Anom .* 1e6
pherr = (DOfpswf.ph.v - phnom)/phnom .* 1e6
\end{lstlisting}

        \begin{lstlisting}[style=output]
SP-FFT errors (ppm):

ferr =

  -19.9996


Aerr =

   -2.8780


pherr =

  -1.5645e+06

PSFE errors:

ferr =

  -1.1368e-10


Aerr =

   3.8924e-07


pherr =

   3.3073e-04

FPSWF errors:

ferr =

  -1.1368e-10


Aerr =

  -3.2951e-07


pherr =

   1.4655e-08

\end{lstlisting} \color{black}
    

\subsubsection*{Harmonically distorted signal.}

\begin{par}
In other cases a harmonic distortion can appear. Suppose a signal with second order harmonic of 10\% amplitude as the main signal.
\end{par} \vspace{1em}
\begin{lstlisting}[style=mcode]
hadist_wave = Anom*sin(2*pi*fnom*timestamps + phnom) + 0.1*Anom*sin(2*pi*fnom*2*timestamps + 2);
DI.y.v = hadist_wave;
DOspfft = qwtb('SP-FFT', DI);
DOpsfe = qwtb('PSFE', DI);
DOfpswf = qwtb('FPSWF', DI);
\end{lstlisting}

        \begin{lstlisting}[style=output]
QWTB: no uncertainty calculation
QWTB: no uncertainty calculation
QWTB: no uncertainty calculation
Fitting started

Local minimum possible.

lsqnonlin stopped because the final change in the sum of squares relative to 
its initial value is less than the default value of the function tolerance.



Fitting finished
\end{lstlisting} \color{black}
    

\subsubsection*{Compare results for harmonically distorted signal.}

\begin{par}
Comparison of relative errors. PSFE or FPSWF are not affected by harmonic distortion, however FPSWF is not suitable.
\end{par} \vspace{1em}
\begin{lstlisting}[style=mcode]
disp('SP-FFT errors (ppm):')
[tmp, ind] = max(DOspfft.A.v);
ferr  = (DOspfft.f.v(ind) - fnom)/fnom .* 1e6
Aerr  = (DOspfft.A.v(ind) - Anom)/Anom .* 1e6
pherr = (DOspfft.ph.v(ind) - phnom)/phnom .* 1e6

disp('PSFE errors:')
ferr  = (DOpsfe.f.v - fnom)/fnom .* 1e6
Aerr  = (DOpsfe.A.v - Anom)/Anom .* 1e6
pherr = (DOpsfe.ph.v - phnom)/phnom .* 1e6

disp('FPSWF errors:')
ferr  = (DOfpswf.f.v - fnom)/fnom .* 1e6
Aerr  = (DOfpswf.A.v - Anom)/Anom .* 1e6
pherr = (DOfpswf.ph.v - phnom)/phnom .* 1e6
\end{lstlisting}

        \begin{lstlisting}[style=output]
SP-FFT errors (ppm):

ferr =

     0


Aerr =

     0


pherr =

  -1.5708e+06

PSFE errors:

ferr =

  -2.2737e-10


Aerr =

   6.7212e-04


pherr =

    0.5311

FPSWF errors:

ferr =

   -0.7356


Aerr =

    0.1407


pherr =

  231.4552

\end{lstlisting} \color{black}
    

\subsubsection*{Harmonically distorted, noisy, non-coherent signal.}

\begin{par}
In final test all distortions are put in a waveform and results are compared.
\end{par} \vspace{1em}
\begin{lstlisting}[style=mcode]
err_wave = Anom*sin(2*pi*fnc*timestamps + phnom) + 0.1*Anom*sin(2*pi*fnc*2*timestamps + 2) + normrnd(0, 100e-6, size(ideal_wave));
DI.y.v = err_wave;
DOspfft = qwtb('SP-FFT', DI);
DOpsfe = qwtb('PSFE', DI);
DOfpswf = qwtb('FPSWF', DI);
\end{lstlisting}

        \begin{lstlisting}[style=output]
QWTB: no uncertainty calculation
QWTB: no uncertainty calculation
QWTB: no uncertainty calculation
Fitting started

Local minimum possible.

lsqnonlin stopped because the final change in the sum of squares relative to 
its initial value is less than the default value of the function tolerance.



Fitting finished
\end{lstlisting} \color{black}
    

\subsubsection*{Compare results for harmonically distorted, noisy, non-coherent signal.}

\begin{lstlisting}[style=mcode]
disp('SP-FFT errors (ppm):')
[tmp, ind] = max(DOspfft.A.v);
ferr  = (DOspfft.f.v(ind) - fnc)/fnc .* 1e6
Aerr  = (DOspfft.A.v(ind) - Anom)/Anom .* 1e6
pherr = (DOspfft.ph.v(ind) - phnom)/phnom .* 1e6

disp('PSFE errors:')
ferr  = (DOpsfe.f.v - fnc)/fnc .* 1e6
Aerr  = (DOpsfe.A.v - Anom)/Anom .* 1e6
pherr = (DOpsfe.ph.v - phnom)/phnom .* 1e6

disp('FPSWF errors:')
ferr  = (DOfpswf.f.v - fnc)/fnc .* 1e6
Aerr  = (DOfpswf.A.v - Anom)/Anom .* 1e6
pherr = (DOfpswf.ph.v - phnom)/phnom .* 1e6
\end{lstlisting}

        \begin{lstlisting}[style=output]
SP-FFT errors (ppm):

ferr =

  -19.9996


Aerr =

   -1.2583


pherr =

  -1.5645e+06

PSFE errors:

ferr =

    0.0111


Aerr =

    1.1518


pherr =

   -3.2602

FPSWF errors:

ferr =

   -0.7098


Aerr =

    1.2924


pherr =

  222.5868

\end{lstlisting} \color{black}
    


%%% \end{document}
    
