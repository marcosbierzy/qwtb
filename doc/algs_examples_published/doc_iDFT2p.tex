
% This LaTeX was auto-generated from an M-file by MATLAB.
% To make changes, update the M-file and republish this document.

%%% \documentclass{article}
%%% \usepackage{graphicx}
%%% \usepackage{color}

%%% \sloppy
%%% \definecolor{lightgray}{gray}{0.5}
\setlength{\parindent}{0pt}

%%% \begin{document}

    
    
\subsection*{2-point interpolated DFT frequency estimator}

\begin{par}
Example for algorithm iDFT3p.
\end{par} \vspace{1em}
\begin{par}
iDFT2p is an algorithm for estimating the frequency, amplitude, and phase of the fundamental component using interpolated discrete Fourier transform. Rectangular or Hann window can be used for DFT.'; See also Krzysztof Duda: Interpolation algorithms of DFT for parameters estimation of sinusoidal and damped sinusoidal signals. In S. M. Salih, editor, Fourier Transform - Signal Processing, chapter 1, pages 3-32, InTech, 2012. \url{http://www.intechopen.com/books/fourier-transform-signal-processing/interpolated-dft} Implemented by Rado Lapuh, 2016.';
\end{par} \vspace{1em}

\subsubsection*{Contents}

\begin{itemize}
\setlength{\itemsep}{-1ex}
   \item Generate sample data
   \item Call algorithm
   \item Display results
\end{itemize}


\subsubsection*{Generate sample data}

\begin{par}
Two quantities are prepared: \lstinline{Ts} and \lstinline{y}, representing 0.5 second of sinus waveform of nominal frequency 100 Hz, nominal amplitude 1 V and nominal phase 1 rad, sampled with sampling time 0.1 ms, with offset 0.1 V. The sampling is not coherent.
\end{par} \vspace{1em}
\begin{lstlisting}[style=mcode]
DI = [];
Anom = 1; fnom = 100; phnom = 1; Onom = 0.1;
DI.Ts.v = 1e-4;
t = [0:DI.Ts.v:0.5];
DI.y.v = Anom*sin(2*pi*fnom*t + phnom) + Onom;
\end{lstlisting}


\subsubsection*{Call algorithm}

\begin{par}
First a rectangular window will be selected to estimate main signal properties. Use QWTB to apply algorithm \lstinline{iDFT3p} to data \lstinline{DI} and put results into \lstinline{DOr}.
\end{par} \vspace{1em}
\begin{lstlisting}[style=mcode]
DI.window.v = 'rectangular';
DOr = qwtb('iDFT2p', DI);
\end{lstlisting}

        \begin{lstlisting}[style=output]
QWTB: no uncertainty calculation
\end{lstlisting} \color{black}
    \begin{par}
Next a Hann window will be selected to estimate main signal properties Results will be put into \lstinline{DOh}.
\end{par} \vspace{1em}
\begin{lstlisting}[style=mcode]
DI.window.v = 'Hann';
DOh = qwtb('iDFT2p', DI);
\end{lstlisting}

        \begin{lstlisting}[style=output]
QWTB: no uncertainty calculation
\end{lstlisting} \color{black}
    

\subsubsection*{Display results}

\begin{par}
Results is the amplitude, frequency and phase of sampled waveform. For rectangular window, the error from nominal in parts per milion is:
\end{par} \vspace{1em}
\begin{lstlisting}[style=mcode]
f_re = (DOr.f.v - fnom)./fnom .* 1e6
A_re = (DOr.A.v - Anom)./Anom .* 1e6
ph_re = (DOr.ph.v - phnom)./phnom .* 1e6
O_re = (DOr.O.v - Onom)./Onom .* 1e6
\end{lstlisting}

        \begin{lstlisting}[style=output]

f_re =

   -0.8083


A_re =

   40.3148


ph_re =

  217.9412


O_re =

   1.6826e+03

\end{lstlisting} \color{black}
    \begin{par}
For Hann window:
\end{par} \vspace{1em}
\begin{lstlisting}[style=mcode]
f_he = (DOh.f.v - fnom)./fnom .* 1e6
A_he = (DOh.A.v - Anom)./Anom .* 1e6
ph_he = (DOh.ph.v - phnom)./phnom .* 1e6
O_he = (DOh.O.v - Onom)./Onom .* 1e6
\end{lstlisting}

        \begin{lstlisting}[style=output]

f_he =

   1.8426e-04


A_he =

   -0.0046


ph_he =

    6.2442


O_he =

   -0.6862

\end{lstlisting} \color{black}
    


%%% \end{document}
    
