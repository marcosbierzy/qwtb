
% This LaTeX was auto-generated from an M-file by MATLAB.
% To make changes, update the M-file and republish this document.

%%% \documentclass{article}
%%% \usepackage{graphicx}
%%% \usepackage{color}

%%% \sloppy
%%% \definecolor{lightgray}{gray}{0.5}
\setlength{\parindent}{0pt}

%%% \begin{document}

    
    
\subsection*{Four parameter sine wave fitting}

\begin{par}
Example for algorithm FPNLSF.
\end{par} \vspace{1em}
\begin{par}
FPNLSF is an algorithm for estimating the frequency, amplitude, and phase of the sine waveform. The algorithm use least squares method. Algorithm requires good estimate of frequency.
\end{par} \vspace{1em}

\subsubsection*{Contents}

\begin{itemize}
\setlength{\itemsep}{-1ex}
   \item Generate sample data
   \item Call algorithm
   \item Display results
\end{itemize}


\subsubsection*{Generate sample data}

\begin{par}
Two quantities are prepared: \lstinline{t} and \lstinline{y}, representing 1 second of sinus waveform of nominal frequency 1 kHz, nominal amplitude 1 V, nominal phase 1 rad and offset 1 V sampled at sampling frequency 10 kHz.
\end{par} \vspace{1em}
\begin{lstlisting}[style=mcode]
DI = [];
Anom = 2; fnom = 100; phnom = 1; Onom = 0.2;
DI.t.v = [0:1/1e4:1-1/1e4];
DI.y.v = Anom*sin(2*pi*fnom*DI.t.v + phnom) + Onom;
\end{lstlisting}
\begin{par}
Lets make an estimate of frequency 0.2 percent higher than nominal value:
\end{par} \vspace{1em}
\begin{lstlisting}[style=mcode]
DI.fest.v = 100.2;
\end{lstlisting}


\subsubsection*{Call algorithm}

\begin{par}
Use QWTB to apply algorithm \lstinline{FPNLSF} to data \lstinline{DI}.
\end{par} \vspace{1em}
\begin{lstlisting}[style=mcode]
CS.verbose = 1;
DO = qwtb('FPNLSF', DI, CS);
\end{lstlisting}

        \begin{lstlisting}[style=output]
QWTB: no uncertainty calculation
Fitting started

Local minimum found.

Optimization completed because the size of the gradient is less than
the default value of the function tolerance.



Fitting finished
\end{lstlisting} \color{black}
    

\subsubsection*{Display results}

\begin{par}
Results is the amplitude, frequency and phase of sampled waveform.
\end{par} \vspace{1em}
\begin{lstlisting}[style=mcode]
A = DO.A.v
f = DO.f.v
ph = DO.ph.v
O = DO.O.v
\end{lstlisting}

        \begin{lstlisting}[style=output]

A =

    2.0000


f =

   100


ph =

    1.0000


O =

    0.2000

\end{lstlisting} \color{black}
    \begin{par}
Errors of estimation in parts per milion:
\end{par} \vspace{1em}
\begin{lstlisting}[style=mcode]
Aerrppm = (DO.A.v - Anom)/Anom .* 1e6
ferrppm = (DO.f.v - fnom)/fnom .* 1e6
pherrppm = (DO.ph.v - phnom)/phnom .* 1e6
Oerrppm = (DO.O.v - Onom)/Onom .* 1e6
\end{lstlisting}

        \begin{lstlisting}[style=output]

Aerrppm =

   4.8894e-07


ferrppm =

     0


pherrppm =

   4.8850e-09


Oerrppm =

  -1.1102e-08

\end{lstlisting} \color{black}
    


%%% \end{document}
    
