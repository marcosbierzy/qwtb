
% This LaTeX was auto-generated from an M-file by MATLAB.
% To make changes, update the M-file and republish this document.

%%% \documentclass{article}
%%% \usepackage{graphicx}
%%% \usepackage{color}

%%% \sloppy
%%% \definecolor{lightgray}{gray}{0.5}
\setlength{\parindent}{0pt}

%%% \begin{document}

    
    
\subsection*{Four parameter sine wave fitting}

\begin{par}
Example for algorithm ThreePSF.
\end{par} \vspace{1em}
\begin{par}
ThreePSF is an algorithm for estimating the amplitude, phase and offset of the sine waveform according standard IEEE Std 1241-2000';
\end{par} \vspace{1em}

\subsubsection*{Contents}

\begin{itemize}
\setlength{\itemsep}{-1ex}
   \item Generate sample data
   \item Call algorithm
   \item Display results
\end{itemize}


\subsubsection*{Generate sample data}

\begin{par}
Two quantities are prepared: \lstinline{t} and \lstinline{y}, representing 1 second of sinus waveform of nominal frequency 1 kHz, nominal amplitude 1 V, nominal phase 1 rad and offset 1 V sampled at sampling frequency 10 kHz.
\end{par} \vspace{1em}
\begin{lstlisting}[style=mcode]
DI = [];
Anom = 2; fnom = 100; phnom = 1; Onom = 0.2;
t = [0:1/1e4:1-1/1e4];
DI.y.v = Anom*sin(2*pi*fnom*t + phnom) + Onom;
DI.Ts.v = 1e-4;
DI.f.v = fnom;
\end{lstlisting}


\subsubsection*{Call algorithm}

\begin{par}
Use QWTB to apply algorithm \lstinline{ThreePSF} to data \lstinline{DI}.
\end{par} \vspace{1em}
\begin{lstlisting}[style=mcode]
CS.verbose = 1;
DO = qwtb('ThreePSF', DI, CS);
\end{lstlisting}

        \begin{lstlisting}[style=output]
QWTB: no uncertainty calculation
\end{lstlisting} \color{black}
    

\subsubsection*{Display results}

\begin{par}
Results is the amplitude, phase and offset of sampled waveform.
\end{par} \vspace{1em}
\begin{lstlisting}[style=mcode]
A = DO.A.v
ph = DO.ph.v
O = DO.O.v
\end{lstlisting}

        \begin{lstlisting}[style=output]

A =

    2.0000


ph =

    1.0000


O =

    0.2000

\end{lstlisting} \color{black}
    \begin{par}
Errors of estimation in parts per milion:
\end{par} \vspace{1em}
\begin{lstlisting}[style=mcode]
Aerrppm = (DO.A.v - Anom)/Anom .* 1e6
pherrppm = (DO.ph.v - phnom)/phnom .* 1e6
Oerrppm = (DO.O.v - Onom)/Onom .* 1e6
\end{lstlisting}

        \begin{lstlisting}[style=output]

Aerrppm =

  -8.4377e-09


pherrppm =

  -9.7700e-09


Oerrppm =

   8.3267e-10

\end{lstlisting} \color{black}
    


%%% \end{document}
    
