
% This LaTeX was auto-generated from an M-file by MATLAB.
% To make changes, update the M-file and republish this document.

%%% \documentclass{article}
%%% \usepackage{graphicx}
%%% \usepackage{color}

%%% \sloppy
%%% \definecolor{lightgray}{gray}{0.5}
\setlength{\parindent}{0pt}

%%% \begin{document}

    
    
\subsection*{Spurious Free Dynamic Range by means of Fast fourier transform}

\begin{par}
Example for algorithm SFDR.
\end{par} \vspace{1em}
\begin{par}
Calculates SFDR by calculating FFT spectrum.
\end{par} \vspace{1em}

\subsubsection*{Contents}

\begin{itemize}
\setlength{\itemsep}{-1ex}
   \item Generate sample data
   \item Call algorithm
   \item Display results
\end{itemize}


\subsubsection*{Generate sample data}

\begin{par}
First quantity \lstinline{y} representing 1 second of signal containing spurious component is prepared. Main signal component has nominal frequency 1 kHz, nominal amplitude 2 V, nominal phase 1 rad and offset 1 V sampled at sampling frequency 10 kHz.
\end{par} \vspace{1em}
\begin{lstlisting}[style=mcode]
DI = [];
fsnom = 1e4; Anom = 4; fnom = 100; phnom = 1; Onom = 0.2;
t = [0:1/fsnom:1-1/fsnom];
DI.y.v = Anom*sin(2*pi*fnom*t + phnom);
\end{lstlisting}
\begin{par}
A spurious component with amplitude at 1/100 of main carrier frequency is added. Thus by definition the SFDR in dBc has to be 40.
\end{par} \vspace{1em}
\begin{lstlisting}[style=mcode]
DI.y.v = DI.y.v + Anom./100*sin(2*pi*fnom*3.5*t + phnom);
\end{lstlisting}


\subsubsection*{Call algorithm}

\begin{par}
Use QWTB to apply algorithm \lstinline{SFDR} to data \lstinline{DI}.
\end{par} \vspace{1em}
\begin{lstlisting}[style=mcode]
DO = qwtb('SFDR', DI);
\end{lstlisting}

        \begin{lstlisting}[style=output]
QWTB: no uncertainty calculation
\end{lstlisting} \color{black}
    

\subsubsection*{Display results}

\begin{par}
Result is the SFDR (dBc).
\end{par} \vspace{1em}
\begin{lstlisting}[style=mcode]
SFDR = DO.SFDRdBc.v
\end{lstlisting}

        \begin{lstlisting}[style=output]

SFDR =

   40.0000

\end{lstlisting} \color{black}
    


%%% \end{document}
    
