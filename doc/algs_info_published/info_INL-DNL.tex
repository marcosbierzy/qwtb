\begin{tightdesc}
\item [Id:] INL-DNL
\item [Name:] Integral and Differential Non-Linearity of ADC
\item [Description:] Calculates Integral and Differential Non-Linearity of an ADC. The histogram of measured data is used to calculate INL and DNL estimators. ADC has to sample a pure sine wave. To estimate all transition levels the amplitude of the sine wave should overdrive the full range of the ADC by at least 120%. If not so, non estimated transition levels will be assumed to be 0 and the results may be less accurate. As an input ADC codes are required.
\item [Citation:] Estimators are based on Tamás Virosztek, MATLAB-based ADC testing with sinusoidal excitation signal (in Hungar- ian), B.Sc. Thesis, 2011. Implementation: Virosztek, T., Pálfi V., Renczes B., Kollár I., Balogh L., Sárhegyi A., Márkus J., Bilau Z. T., ADCTest project site: \url{http://www.mit.bme.hu/projects/adctest} 2000-2014
\item [Remarks:] Based on the ADCTest Toolbox v4.3, November 25, 2014.
\item [License:] UNKNOWN
\item [Provides GUF:] no
\item [Provides MCM:] no
\item [Input Quantities] \rule{0em}{0em}
    \begin{tightdesc}
    \item [Required:] 
        \textsf{bitres},\enspace \textsf{codes}
    \end{tightdesc}
\item [Descriptions:] \rule{0em}{0em}
    \begin{tightdesc}
        \item[\textsf{bitres}] -- Bit resolution of an ADC
        \item[\textsf{codes}] -- Sampled values represented as ADC codes (not converted to voltage)
    \end{tightdesc}
\item [Output Quantities] \rule{0em}{0em}
    \begin{tightdesc}
        \item[\textsf{DNL}] -- Differential Non-Linearity
        \item[\textsf{INL}] -- Integral Non-Linearity
    \end{tightdesc}
\end{tightdesc}
